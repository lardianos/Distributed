% Writed by: Stavros Papantonakis
%
%!TEX TS-program = xelatex
%!TEX encoding = UTF-8 Unicode
%
\setcounter{section}{0}
\section{Απάντηση}

\noindent
Αρχικά χρειαζόμαστε 5 αρχεία για να υλοποιήσουμε την εφαρμογή μας
HRInterface.java, HRImpl.java, HRServer.java, HRListener.java, HRClient.java.
\subsection{HRInterface.java}
\noindent 
Σε αυτό το αρχείο δημιουργούμε ένα interface οπού κάνει extends το java.rmi.Remote. Μέσα στο interface δηλώνουμε της μεθόδους που θα μπορεί
 να καλέσει ο Client για να της εκτελέσει ο Server. όλες η μέθοδοι κάνουν
  throws το java.rmi.RemoteException. Εμείς έχουμε δηλώσει 5 μεθόδους.
\begin{itemize}
	\item String[] prit\_list()
	\item String[] book\_reservations(String Type,String num, String Name)
	\item Hashtable<Integer, String[]> print\_guests\_list()
	\item Hashtable<Integer, String[]> cancel\_book(String Type,String num, String Name)
	\item void add\_book\_listener(HRListener hrlistener, String Type)
\end{itemize}

\subsection{HRImpl.java}
\noindent 
Σε αυτό το αρχείο υλοποιούμε όλες της μεθόδους που δηλώσαμε στο HRInterface.java. Αρχικά χρησιμοποιήσουμε 5 μεταβλητές για τον αριθμό δωματίων και άλλες 5 για το κόστος δωματίων, χρειαζόμαστε επίσης και 5 λίστες
μια για κάθε τύπο δωματίου στης οποίες θα εγγράφονται η client που επιθυμούν
να ενημερωθούν σε περίπτωση που αδειάσουν δωμάτια, τέλος χρησιμοποιούμε ένα
Hashtable για να αποθηκεύσουμε της κρατήσεις. 

\noindent
Στον κατασκευαστή αρχικοποιούμε όλες τις τιμές που χρειαζόμαστε και καλούμε
και τον κατασκευαστή της υπερκλάσης super().

\noindent
κάνουμε override την μέθοδο run η οποία καλείτε από το thread του server.
Aυτή η μέθοδος ελέγχει αν η λίστες δεν είναι γεμάτες και αν τα δωμάτια δεν
είναι 0 και καλή την notifyHRListeners με αντίστοιχη παράμετρο τύπου δωματίου
για να ενημέρωση τους εκάστοτε clients.

\noindent
Στην μέθοδο prit\_list() δημιουργούμε ένα String Array με το ποσά δωμάτια
έχουμε και την επιστρέφουμε έτσι ώστε να μπορεί να την τυπώσει ο client.

\noindent
Στην μέθοδο book\_reservations περνούμε ως είσοδο τρία Strings τα String Type,String num, String Name και ελέγχοντας το Type υπολογίζουμε το κόστος αναγράφουμε
την εγγραφή στο Hashtable και επιστρέφουμε στον client ένα String Array με
την εγγραφή και το κόστος. Σε περίπτωση που δεν υπάρχουν αρκετά δωμάτια
του επιστρέφουμε τον αριθμό δωματίου και αν θέλει κάνει κράτηση με αυτά τα
δωμάτια.

Στην μέθοδο print\_guests\_list τυπώνουμε τα περιεχόμενα του επιστρέφουμε
τον Hashtable στον client έτσι ώστε να τύπωση τα περιεχόμενα του.

Στην μέθοδο cancel\_book περνούμε ως είσοδο πάλι τα String Type,String num, String Name και εντοπίζουμε αν υπάρχει κράτηση με αυτό το όνομα και τύπο
δωματίου. Αν υπαρχή ελέγχουμε αν έχει τόσα δωμάτια, τέλος ακυρώνουμε 
αυτόν τον αριθμό δωματίων και επιστρέφουμε ένα hashtable με τα δωμάτια. 

(\textbf{Λόγω χρόνου δεν προλαβαίνω να τελειώσω το documentation})

\subsection{HRServer.java}

\subsection{HRListener.java}

\subsection{HRClient.java}


